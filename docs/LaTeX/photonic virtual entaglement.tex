\documentclass[12pt, letterpaper]{article}
\usepackage[utf8]{inputenc}
\usepackage{amsmath}
\usepackage{amssymb}
\usepackage{physics}
\usepackage{graphicx}
\usepackage{geometry}
\geometry{margin=1in}

\title{\textbf{The Mimetic Twin Protocol: \\ Inducing Photonic Entanglement via Recursive Opto-Electronic Feedback Loops in Commodity Hardware}}
\author{A. Researcher \& H. Coherences \\ \textit{Harpia Coherences Laboratory}}
\date{\today}

\begin{document}

\maketitle

\begin{abstract}
We propose a novel method for generating entropic coherence on standard computational devices without specialized quantum hardware. By establishing a recursive feedback loop between the Digital-to-Analog Converter (GPU/Display) and the Analog-to-Digital Converter (Webcam/Sensor), we create a self-referential photonic field. We demonstrate that pointing the sensor at its own display output creates a \textit{Mimetic Twin} state—a bounded region of non-collapsing superposition used to drive the stochastic kernel of \texttt{q-os}.
\end{abstract}

\section{Introduction}
Standard operating systems rely on pseudo-random number generators (PRNGs) which are deterministic. \texttt{q-os} requires a fundamental source of indeterminacy and coherence. We propose that the interface between a display (photon emitter) and a sensor (photon receiver) constitutes a quantum-critical channel when arranged in a recursive loop.

\section{The Opto-Electronic Loop}
The physical configuration requires a "closed-loop" visualization, often described as an "infinite mirror" effect. The system must display the sensor's input directly onto the photon emitter (screen) in real-time.

Let $W_{view}$ be a window on the display rendering the live feed from the webcam.
The Graphics Processing Unit (GPU) converts this digital buffer into an analog photon stream:
\begin{equation}
    \Phi_{emit}(t) = D_{AC}(W_{view}) \cdot \eta_{screen}
\end{equation}
Where $\Phi_{emit}$ is the photon flux and $\eta_{screen}$ is the quantum efficiency of the display matrix.

The camera, physically positioned to face $W_{view}$, captures this emission. This creates a recursive feedback loop where the screen displays the camera seeing the screen, amplifying microscopic noise into macroscopic chaotic patterns.

\section{Recursive Mimetic Function}
The loop is defined by the re-entry of the signal into the frame buffer.
\begin{equation}
    B_{t+1} = A_{DC} \left( P_{prop} \left( D_{AC}(B_t) \right) + \xi_{env} \right)
\end{equation}
Where $P_{prop}$ represents the propagation of photons through the free space gap $\Delta x$, and $\xi_{env}$ represents the environmental noise (photon shot noise, thermal fluctuations).

Because the camera is filming its own output, any microscopic fluctuation in $\xi_{env}$ is amplified exponentially with each frame $t$. We define the **Mimetic Twin** $\ket{\mathcal{M}}$ not as the image itself, but as the \textit{divergence} between the digital expectation and the analog reality:
\begin{equation}
    \ket{\mathcal{M}} = B_{t+1} - \hat{U} B_t
\end{equation}
Here, $\ket{\mathcal{M}}$ represents the pure quantum noise harvested from the photon flight—the "entanglement" of the digital logic with the physical world.

\section{Integration with q-os}
In the \texttt{q-os} architecture, this Mimetic Twin is not discarded as video feedback noise. It is used to seed the \textit{Mimetic Processing Unit} (MPU).

The Operating System treats the screen buffer not as output, but as an entanglement surface.
\begin{equation}
    S_{q-os} = \int_{0}^{T} \ket{\mathcal{M}(t)} dt
\end{equation}
This integral allows \texttt{q-os} to maintain a "Non-Collapsing Superposition" by constantly re-injecting the analog state into the digital memory, preventing the system from decohering into pure binary determinism.

\section{Conclusion}
By utilizing the GPU-Screen-Webcam loop, we convert a standard laptop into a closed-loop photonic reactor. This allows \texttt{q-os} to access a continuous stream of Mimetic Entanglement, bridging the gap between binary computation and organic field dynamics.

\end{document}